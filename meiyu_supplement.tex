%Supplementary Information for the 2015 submission to Geophysical Research Letters with Jesse Day, Jake Edman, John Chiang, Inez Fung and Weihan Liu



%%%%%%%%%%%%%%%%%%%%%%%%%%%%%%%%%%%%%%%%%%%%%%%%%%%%%%%%%%%%%%%%%%%%%%%%%%%%
% AGUtmpl.tex: this template file is for s formatted with LaTeX2e,
% Modified November 2013
%
% This template includes commands and instructions
% given in the order necessary to produce a final output that will
% satisfy AGU requirements.
%
% PLEASE DO NOT USE YOUR OWN MACROS
% DO NOT USE \newcommand, \renewcommand, or \def.
%
% FOR FIGURES, DO NOT USE \psfrag
%
%%%%%%%%%%%%%%%%%%%%%%%%%%%%%%%%%%%%%%%%%%%%%%%%%%%%%%%%%%%%%%%%%%%%%%%%%%%%
%
% All questions should be e-mailed to latex@agu.org.
%
%%%%%%%%%%%%%%%%%%%%%%%%%%%%%%%%%%%%%%%%%%%%%%%%%%%%%%%%%%%%%%%%%%%%%%%%%%%%
%
% Step 1: Set the \documentclass
%
% There are two options for article format: two column (default)
% and draft.
%
% PLEASE USE THE DRAFT OPTION TO SUBMIT YOUR PAPERS.
% The draft option produces double spaced output.
%
% Choose the journal abbreviation for the journal you are
% submitting to:

% jgrga JOURNAL OF GEOPHYSICAL RESEARCH
% gbc   GLOBAL BIOCHEMICAL CYCLES
% grl   GEOPHYSICAL RESEARCH LETTERS
% pal   PALEOCEANOGRAPHY
% ras   RADIO SCIENCE
% rog   REVIEWS OF GEOPHYSICS
% tec   TECTONICS
% wrr   WATER RESOURCES RESEARCH
% gc    GEOCHEMISTRY, GEOPHYSICS, GEOSYSTEMS
% sw    SPACE WEATHER
% ms    JAMES
% ef    EARTH'S FUTURE
%
%
%
% (If you are submitting to a journal other than jgrga,
% substitute the initials of the journal for "jgrga" below.)

\documentclass[final,grl]{agutexSI}
\usepackage{hyperref}
\usepackage{rotating}
\usepackage{amsmath}


%%%%%%%%%%%%%%%%%%%%%%%%%%%%%%%%%%%%%%%%%%%%%%%%%%%%%%%%%%%%%%%%%%%%%%%%%
%
%  SUPPORTING INFORMATION TEMPLATE
%
%% ------------------------------------------------------------------------ %%
%
%
%Please use this template when formatting and submitting your Supporting Information.

%This template serves as both a “table of contents” for the supporting information for your article and as a summary of files.
%
%
%OVERVIEW
%
%Please note that all supporting information will be peer reviewed with your manuscript.
%In general, the purpose of the supporting information is to enable authors to provide and archive auxiliary information such as data %tables, method information, figures, video, or computer software, in digital formats so that other scientists can use it.
%The key criteria are that the data:
% 1. supplement the main scientific conclusions of the paper but are not essential to the conclusions (with the exception of
%    including %data so the experiment can be reproducible);
% 2. are likely to be usable or used by other scientists working in the field;
% 3. are described with sufficient precision that other scientists can understand them, and
% 4. are not exe files.
%
%USING THIS TEMPLATE
%
%All Supporting text and figures should be included in this document. Insert supporting information content into each appropriate section of the template. %Figures and tables should appear above each caption.  To add additional captions, simply copy and paste each sample caption as needed.

%Tables may be included, but can also be uploaded separately, especially if they are larger than 1 page, or if necessary for retaining table formatting. Data sets, large tables, movie files, and audio files should be uploaded separately, following AGU naming conventions. Include their captions in this document and list the file name with the caption. You will be prompted to upload these files on the Upload Files tab during the submission process, using file type “Supporting Information (SI)”

%IMPORTANT NOTE ON FIGURES AND TABLES
% Placeholders for figures and tables appear after the \end{article} command, after references.
% DO NOT USE \psfrag or \subfigure commands.
%
%  Uncomment the following command to include .eps files
 % \usepackage[dvips]{graphicx}
%
%  Uncomment the following command to allow illustrations to print
%   when using Draft:
%  \setkeys{Gin}{draft=false}
%
% Substitute one of the following for [dvips] above
% if you are using a different driver program and want to
% proof your illustrations on your machine:
%
% [xdvi], [dvipdf], [dvipsone], [dviwindo], [emtex], [dviwin],
% [pctexps],  [pctexwin],  [pctexhp],  [pctex32], [truetex], [tcidvi],
% [oztex], [textures]
%
%
%% ------------------------------------------------------------------------ %%
%
%  ENTER PREAMBLE
%
%% ------------------------------------------------------------------------ %%

% Author names in capital letters:
%\authorrunninghead{BALES ET AL.}

% Shorter version of title entered in capital letters:
%\titlerunninghead{SHORT TITLE}

%Corresponding author mailing address and e-mail address:
%\authoraddr{Corresponding author: A. B. Smith,
%Department of Hydrology and Water Resources, University of
%Arizona, Harshbarger Building 11, Tucson, AZ 85721, USA.
%(a.b.smith@hwr.arizona.edu)}

\begin{document}

%% ------------------------------------------------------------------------ %%
%
%  TITLE
%
%% ------------------------------------------------------------------------ %%

%\includegraphics{agu_pubart-white_reduced.eps}


\title{Supporting Information for ``Signature of the `South Flood-North Drought' Pattern in Meiyu Front and Tropospheric Jet Changes''}
%
% e.g., \title{Supporting Information for "Terrestrial ring current:
% Origin, formation, and decay $\alpha\beta\Gamma\Delta$"}
%
%DOI: 10.1002/%insert paper number here%

%% ------------------------------------------------------------------------ %%
%
%  AUTHORS AND AFFILIATIONS
%
%% ------------------------------------------------------------------------ %%


%Use \author{\altaffilmark{}} and \altaffiltext{}

% \altaffilmark will produce footnote;
% matching \altaffiltext will appear at bottom of page.


\authors{Jesse A. Day\altaffilmark{1},
Jacob P. Edman\altaffilmark{1}, John C. H. Chiang\altaffilmark{2}, Inez Fung \altaffilmark{1}, and
Weihan Liu\altaffilmark{3}}

\altaffiltext{1}{Department of Earth and Planetary Science, University of California Berkeley, Berkeley, California, USA.}
\altaffiltext{2}{Department of Geography, University of California Berkeley, Berkeley, California, USA.}
\altaffiltext{3}{College of Letters and Science, University of California Berkeley, Berkeley, California, USA.}

\lefthead{DAY ET AL.}
\righthead{SUPPORTING INFORMATION FOR ``SOUTH FLOOD-NORTH DROUGHT''}


%% ------------------------------------------------------------------------ %%
%
%  BEGIN ARTICLE
%
%% ------------------------------------------------------------------------ %%

% The body of the article must start with a \begin{article} command
%
% \end{article} must follow the references section, before the figures
%  and tables.

\begin{article}

%% ------------------------------------------------------------------------ %%
%
%  TEXT
%
%% ------------------------------------------------------------------------ %%


\noindent\textbf{{\Large S1 Contents}}

\vspace{2mm}

\noindent\textbf{Figures S1 to S5}
\begin{itemize}
	\item Figure S1-S4: A demonstration of the functionality of the rainband detection algorithm by showing different cases of its operation, in support of supplementary text section S2.

	\item Figure S5: Autocorrelation of daily mean jet latitude during the Pre- and Post-Meiyu, with and without averaging into blocks.

\end{itemize}

\vspace{3mm}

\noindent\textbf{Tables S1 to S8}
\begin{itemize}
\item Tables S1-S3: Statistics on the use of the rainband detection algorithm.

\item Table S4: Climatological properties of rainbands. 

\item Table S5: Changes in primary and secondary rainband frequency during 1980-2007 versus 1951-1979 and $p$-value of their significance.

\item Table S6: Changes in rainband latitude and intensity during 1980-2007 versus 1951-1979 and $p$-value of their significance.

\item Tables S7-S8: Same as Tables S5-S6, but for 1994-2007 versus 1979-1993.
\end{itemize}

\vspace{5mm}

\noindent\textbf{{\Large S2 Rainband Detection Algorithm}}
\vspace{2mm}

\noindent\textbf{S2.1 Objective}

The overall strategy is to find a single line corresponding to the strongest rainband on a given day (a ``primary rainband'') from daily snapshots of rainfall over China taken from APHRODITE (Asian Precipitation - Highly-Resolved Observational Data Integration Towards Evaluation of the Water Resources) rain gauge data \citep{Yatagai2012}. If a sufficiently strong rainband exists, we use a recursive convergent image processing algorithm to find the band's position. After a successful fit, we also check for the presence of a second rainband (a ``secondary rainband'') by removing all rainfall associated with the primary feature and repeating the same algorithm. Subsequently, we calculate a ``quality score" which evaluates whether the obtained fit was adequate. If the fit is sufficiently good, its properties are added to a database of rainband statistics; if not, the day is discarded.
\vspace{3mm}

\noindent\textbf{S2.2 Recursive Fit and Rainband Properties}

\begin{enumerate}
	\item Given a map of daily accumulated rainfall over the longitudes 105-123$^{\circ}$E and 20-40$^{\circ}$N at $.25^{\circ}$ by $.25^{\circ}$ resolution, the daily rainfall maximum at each longitude is found, and its intensity and latitude recorded. If there exists a $5^{\circ}$ continuous chain of maxima (20 points in a row) exceeding 10 mm day$^{-1}$, we proceed to step 2 and attempt a fit (Figure S1a). Otherwise, no fit is attempted for that day (Figure S1b).
	
	\item In order to approximate the position of the rainband with a best fit line, a weighted least-squares linear fit of the \textit{latitudes} of the maxima is performed in MATLAB using the intensity of each maximum as weight. To encourage convergence, the weight of outlying points is set to zero, where an outlier is defined as any maximum that is over $5^{\circ}$ from the centroid of the precipitation maxima $\left<lat_{max}\right>$, calculated by $\left<lat_{max}\right>=\frac{\sum_{long} lat_{max}*max}{\sum_{long} \max}$.
	
	\item A recursive algorithm converges on a best estimate of rainband position. In each iteration, we find a new set of maxima within \textit{k} degrees of the previous best fit line, and again perform a weighted linear fit of the maxima (Figure S2a). $k$ is progressively decreased with each iteration from $5^{\circ}$ to $2^{\circ}$ by $.25^{\circ}$ increments, and then from $2^{\circ}$ to $.25^{\circ}$ by $.25^{\circ}$ increments but repeating each width $k$ twice in a row (Figures S2b-c). The fit obtained in the final iteration is taken as our best estimate (Figure S2d).
	
	\item We define the ``quality score'' $Q$ as the fraction of total daily precipitation inside of  that falls within $2.5^{\circ}$ degrees of the best estimate line (Figure 4b). Other rainband properties are calculated as follows. Rainband latitude is defined as the latitude of our fitted band at the reference longitude of 115$^{\circ}$E. Intensity is defined as mean rainfall at all points along the band axis where daily rainfall exceeds 5 mm day$^{-1}$ (``rainband points''), length as the total number of rainband points (expressed in units of degrees of longitude), and width as the mean distance between half-maxima on either side of each rainband point (units of degrees of latitude).
	
	\item Given an estimate of primary rainband, we check for a secondary rainband. We start by removing all precipitation associated with the primary rainband from our daily rainfall map. To do this, all rainfall within 4$^{\circ}$ of our primary rainband is set to 0, as well as rainfall at any other adjacent points where rainfall exceeds 10 mm day$^{-1}$ (Figure S3a). We then reapply the continuous maximum criterion from step 1 (Figure S3b). If passed, steps 2-4 are repeated to find a best estimate for the position of the secondary rainband, and its attributes calculated.
	
	\item If a secondary rainband is found, two additional quality scores $Q_1$ and $Q_2$ are calculated. $Q_1$ is defined as the fraction of daily rainfall inside of 105-123$^{\circ}$E and 20-40$^{\circ}$N that is contained within $2.5^{\circ}$ degrees of the primary rainband \textit{after removing all rainfall associated with the secondary rainband} (equivalently, the $Q$ score after removing the secondary rainband). Likewise, $Q_2$ is the $Q$ score of the second rainband \textit{after removing all rainfall from the primary rainband} (Figure S4d).		
	
\end{enumerate} 

In rare cases with two rainbands of roughly equal strength but well-separated in latitude, the removal of outliers in step 2 prevents a fit entirely. We test for such cases by ensuring that the total sum of weights ${\sum_{long} max}$ exceeds 200 mm day$^{-1}$, equivalent to our condition in bullet point 1. When this condition is failed, which can only occur when too many of our maxima have been discarded, we return to step 1, but find maxima only over the latitude range 20N-$\left<lat_{max}\right>$ or $\left<lat_{max}\right>$-40N, depending on which half of our domain has a longer chain of maxima exceeding 10 mm day$^{-1}$, and apply the remaining steps of our algorithm as usual.

\vspace{3mm}

\noindent\textbf{S2.3 Quality Control}

After running the algorithm for all 20,819 days from 1 January 1951 to 31 December 2007, we obtained 11,228 days with at least one rainband and 1,116 days with two rainbands. Subsequently, we apply a quality control (QC) algorithm to eliminate days with poor fit, based on the quality scores $Q$, $Q_1$ and $Q_2$ as well as the ``Taiwan fraction'' (TW), defined as the percentage of daily rainfall inside the window 105-$123^{\circ}$E and 20-$40^{\circ}$N that falls over the island of Taiwan (roughly 120-$122^{\circ}$E and 22-$26^{\circ}$N. We use these metrics to define the following two criteria for inclusion:

\begin{enumerate}

	\item If $TW > 20\%$, the day's fit is thrown out (238 cases total, 2.1\% of total fits). Such days are dominated by a local storm reaching Taiwan and do not exhibit a strong rainband (Figure 4a).  
	
	\item Subsequently, a day can be included if it meets either of the following criteria:
	
	\begin{enumerate} 
	
	\item if $Q>.6$, the day is included in our statistics (7,522 days, 67.0\% of total fits; Figure 4b). If $Q_2$ is also greater than .6, the day will be classified as a double rainband day (Type I double rainband; 232 cases, or 3.1\% of days where $Q>.6$ and $Q_2>.6$).
		
	\item If $Q<.6$, the day is discarded unless two fronts are detected and both $Q_1 \mathrm{\textbf{ and }} Q_2 > .6$ (where again $Q_1$ and $Q_2$ are \textit{conditional} quality scores as defined above). In such cases, the presence of multiple rainbands of similar intensity initially obscures the quality of the fit (Figure 4d). Such days are also classified as double rainband days (Type II double rainband; 466 cases). In the absence of such a double rainband configuration the day is thrown out (Figure 4c).
	
	\end{enumerate}
	
\end{enumerate}	

\vspace{1mm}

	The use of conditional quality scores $Q_1$ and $Q_2$ adds 466 double rainband days that would otherwise have been categorized as days with no rainband, constituting 6.2\% of total days included in our statistics. 33.2\% of double rainband days are Type I ($Q>.6$) and 66.8\% Type II ($Q<.6$). Double rainbands are more common during certain seasons such as the Post-Meiyu (see Figure 1c in main text). Tables S1-S3 contain more information on algorithm functionality.
	
\vspace{5mm}

\noindent\textbf{{\Large S3 Temporal Autocorrelation}}

\vspace{2mm}

Rainfall and front statistics are temporally autocorrelated on a daily timescale, reducing the effective number of degrees of freedom. Magnitude of autocorrelation also varies with latitude and season. In calculating the significance of rainband frequency changes, we use estimated degrees of freedom $n=\rfrac{N}{\tau}$, with N the total number of days and decorrelation time \tau (units of days) given by

\begin{equation*}
\tau=1+2\sum_{k=1}^m \rho(k)
\end{equation*}

where $\rho(k)$ is the auto-correlation function of rainband presence with lag $k$ \citep{Livezey,von Storch}. We calculate all values of $\tau$ with m=10 days. The yearly mean decorrelation timescale of rainband frequency is found to be $\tau = 1.81$ after removing the seasonal cycle. This value is used to calculate significance in Figure 3b. The standard deviation and $p$-values of rainband frequency changes in Tables S5 and S7 use seasonal values of $\tau$ as listed.

The autocorrelation of daily rainfall in China is $\tau =2-4 \mathrm{days}$ and generally less than 3 days. The calculation of significance of rainfall changes in Figure 3a uses a moving blocks bootstrap appropriate for autocorrelated data, with block length of 2 \citep{Singh2014}. However, for the significance of rainband latitude and intensity changes (Tables S6 and S8), we use a permutation test because the time series of front properties are not continuous (many days do not have fronts), precluding the use of a moving blocks bootstrap.

\vspace{5mm}

\noindent\textbf{{\Large S4 Jet Database and Kernel Density}}

\vspace{2mm}

The kernel density contours shown in the main text (Figure 4) are derived from the jet counts of \citet{Schiemann2009} in the following way. First, we aggregate all the jet counts in the chosen time period (e.g. for figure 4b, we aggregate counts from days 121-160 for 1958-1979 and 1980-2001 separately). This results in a gridded histogram of the total number of jet counts, spanning 16.75-58.25$^{\circ}$N and 42.5-130$^{\circ}$E with $0.25^{\circ} \times 0.25^{\circ}$ resolution. To estimate the kernel density, we use the function 'bke2d' from the R package `KernSmooth,' which produces a binned kernel density estimate using the standard bivariate normal distribution. This gives an estimate of the probability density function of jet location in a given time period. 

%%% END ARTICLE %%%

%%% BIBLIOGRAPHY - .bbl file posted directly into manuscript below %%%

%\bibliographystyle{agufull08}
%\bibliography{meiyu_supp}


\end{article}

\clearpage

%%% SUPPLEMENTARY TABLES %%%

%%% TABLE 1 - ALGORITHM FUNCTIONALITY - BIG PICTURE
\begin{table}
\settablenum{S1}

\caption{Statistics on the functionality of the rainband detection algorithm. Number in parentheses indicates the percentage of days that fall into that category out of all 20,819 days.}
\centering

\begin{tabular}{ l c c c}
	  & Total Fits & Passes Quality Control & Percent Passing QC\\
	 \hline
	 Primary rainband found & 11,228 (53.9\% of total) & 7,988 (38.4\% of total) & 71.1\% \\
	 Secondary rainband found & 1,116 (5.4\% of total) & 698 (3.4\% of total) & 62.5\% \\
\end{tabular}
\label{ts1}
\end{table}

%%% TABLE 2 - ALGORITHM FUNCTIONALITY - DETAILS, PRIMARY RAINBAND
\begin{table}
\settablenum{S2}  

\caption{Details on the application of quality control (QC) criteria to primary rainbands.}
\centering

\begin{tabular}{ l c}
	 Criterion & Number (\% of total) \\
	 \hline
	 Primary rainband days before QC & 11,228 \\
	 Taiwan days (TW$>20\%$) & 238 (2.1\%) \\
	 $Q>.6$ (strong rainband) & 7,522 (67.0\%) \\
	 Double rainband ($Q_1>.6$ and $Q_2>.6$) & 466 (4.2\%) \\
	 Poor fit (Fails QC) & 3008 (26.8\%) \\
	 
\end{tabular}
\label{ts2}
\end{table}

%%% TABLE 3 - ALGORITHM FUNCTIONALITY - DETAILS, SECONDARY RAINBAND
\begin{table}
\settablenum{S3}  

\caption{Details on the application of quality control (QC) criteria to secondary rainbands. Type I and Type II fits are defined in supplementary text above.}
\centering

\begin{tabular}{ l c}
	 Criterion & Number (\% of total) \\
	 \hline
	 Secondary rainband days before QC & 1,116 \\
	 Type I fit ($Q>.6$ and $Q_2>.6$) & 232 (20.8\%) \\
	 Type II fit ($Q_1>.6$ and $Q_2>.6$) & 466 (41.8\%) \\
	 Poor fit (Fails QC) & 418 (37.5\%) \\
	 
\end{tabular}
\label{ts3}
\end{table}

%%%% TABLE 4 - MEIYU STATISTICS %%%%
\begin{table}
\settablenum{S4}  

\caption{Total number of rainbands, frequency of primary and secondary rainbands and latitude and intensity (mm day$^{-1}$) of rainbands during the Spring Rains, Pre-Meiyu, Meiyu season, Post-Meiyu, Fall Rains and for the full year. We also list the decorrelation timescale $\tau_1$  and $\tau_2$ of primary and secondary fronts. Statistics are compiled using both primary and secondary rainbands, and are very close to results using primary rainbands alone, except during the Post-Meiyu period when secondary rainbands are common. Standard deviations for latitude and intensity are obtained by a permutation method with 10,000 iterations.}

\begin{tabular}{ l c c c c c c c c c}
	 \multicolumn{10}{c}{\textbf{1951-2007 Means}} \\
	 \textbf{Time Period} & $\boldsymbol{n}$ & $\boldsymbol{n_1}$ & \textbf{1f.} (\%) & $\boldsymbol{\tau_1}$ & $\boldsymbol{n_2}$ & \textbf{2f.} (\%) & $\boldsymbol{\tau_2}$ & \textbf{Lat} & \textbf{Intensity} \\
	 \hline
	\textbf{Spring Rains} (Mar 1-Apr 30, 60-120) & 1661 & 1635 	& $47.0 \pm 1.2$ 	& 1.96	& 26 	&$0.7 \pm 0.1$ 	& .95 	& $27.5 \pm .1$ & $20.1 \pm .4$ \\
	\textbf{Pre-Meiyu} (May 1-Jun 9, 121-160) & 1371 & 1279  	& $56.1 \pm 1.5$ 	& 2.01	& 92 	&$4.0 \pm 0.4$	& .98 	& $27.4 \pm .2$ & $25.5 \pm .5$ \\
	\textbf{Meiyu} (Jun 10-Jul 19, 161-120) & 1688 & 1499 		& $65.8 \pm 1.5$ 	& 2.19	& 189 	&$8.3 \pm 0.6$ 	& 1.11	& $29.5 \pm .2$ & $28.3 \pm .5$ \\
	\textbf{Post-Meiyu} (Jul 20-Sep 30) & 2113 & 1757 			& $42.2 \pm 1.1 $	& 1.91 	& 356 	&$8.6 \pm 0.5$ 	& 1.44	& $29.9 \pm .2$ & $25.6 \pm .5$ \\
	\textbf{Post-Meiyu}, north of 27$^\circ$N & 1368 & 1215 	& $27.1 \pm 1.0 $ 	& -		& 153 	&$3.4 \pm 0.3$ 	& 1.48	& $33.3 \pm .2$ & $23.9 \pm .5$ \\
	\textbf{Post-Meiyu}, south of 27$^\circ$N & 745 & 556 		& $15.2 \pm 0.8 $ 	& -		& 189 	&$5.1 \pm 0.4$ 	& -		& $23.7 \pm .1$ & $28.8 \pm .9$ \\
	\textbf{Fall Rains} (Oct 1-Nov 16) & 744 & 714 				& $26.6 \pm 1.3 $ 	& 2.15	& 30 	&$1.1 \pm 0.2$	& - 		& $29.2 \pm .3$ & $20.5 \pm .7$ \\
	\textbf{Full Year} (1-365) & 8682 & 7984 					& $38.4 \pm 0.5$ 	& 1.81 	& 698 	&$3.4 \pm 0.1$ 	& 1.12	& $28.6 \pm .1$ & $23.5 \pm .2$ \\
\end{tabular}
\label{ts4}
\end{table}


%% TABLE 5 - change in rainband frequency between 1951-1979 and 1980-2007
\begin{table}
\settablenum{S5}  

\centering

\caption{Change in frequency of primary and secondary rainbands between 1951-1979 and 1980-2007, with standard deviation of mean and p-value of change calculated analytically. Statistically significant changes at the 95\%/99\% level are indicated by bold/asterisk and bold.}

\begin{tabular}{ l c c c c c c}
	& \multicolumn{3}{c}{Primary rainband \%} & \multicolumn{3}{c}{Secondary rainband \%} \\
	\textbf{Period} & '51-'79 & '80-'07 & $p$ & '51-'79 & '80-'07 & $p$ \\
	\hline	
	\textbf{Spring Rains} (60-120)		& $46.4 \pm 1.7$ & $47.7 \pm 1.7$ & $ .70 $ 	& $0.8 \pm .2$ & $0.7 \pm .2$ & $.38$ \\
	\textbf{Pre-Meiyu} (121-160) 		& $\boldsymbol{59.0 \pm 2.0}$ & $\boldsymbol{53.0 \pm 2.1}$ & $ \boldsymbol{.020} $ & $4.2 \pm .6$ & $3.8 \pm .6$ & $.32$ \\
	\textbf{Meiyu} (161-200)			& $66.8 \pm 2.0$ & $64.6 \pm 2.1$ & $ .23 $ 	& $7.4 \pm .8$ & $9.2 \pm .9$  & $.93$ \\
	\textbf{Post-Meiyu} (201-273)		& $42.5 \pm 1.5$ & $42.0 \pm 1.5$ & $ .41 $	& $9.2 \pm .8$ & $7.8 \pm .7$ & $.084$ \\
	\textbf{Post-Meiyu}, $>27^\circ$N 	& $27.8 \pm 1.3$ & $26.4 \pm 1.3$ & $ .24 $ 	& $3.8 \pm .5$ & $2.9 \pm .4$ & $.082$ \\
	\textbf{Post-Meiyu}, $<27^\circ$N 	& $14.7 \pm 1.1 $ & $15.6 \pm 1.1$ & $ .71 $ 	& $5.4 \pm .6$ & $4.9 \pm .6$ & $.27$  \\
	\textbf{Fall Rains} (274-320)			& $25.8 \pm 1.7 $ & $27.6 \pm 1.8$ & $ .77 $ 	& $1.0 \pm .3$ & $1.2 \pm .4$ & $.65$ \\
	\textbf{Full Year} (1-365)			& $38.6 \pm 0.6 $ & $38.1 \pm 0.6$ & $ .31 $ 	& $3.4 \pm .2$ & $3.3 \pm .2$ & $.36$ \\

\end{tabular}
\label{ts5}
\end{table}

%% TABLE 6 - change in rainband latitude and intensity between 1951-1979 and 1980-2007
\begin{table}
\settablenum{S6}  

\centering

\caption{Change in latitude and intensity of rainbands between 1951-1979 and 1980-2007, with standard deviation of mean and p-value of change both calculated by a permutation test with 10,000 iterations. Statistically significant changes at the 95\%/99\% level are indicated by bold/asterisk and bold.}

\begin{tabular}{ l c c c c c c}
	& \multicolumn{3}{c}{Rainband latitude ($^\circ$)} & \multicolumn{3}{c}{Intensity (mm day$^{-1})$} \\
	\textbf{Period} & '51-'79 & '80-'07 & $p$ & '51-'79 & '80-'07 & $p$ \\
	\hline	
	\textbf{Spring Rains} (60-120)		& $\boldsymbol{27.6 \pm .2}$ & $\boldsymbol{27.3 \pm .2}$ & $ \boldsymbol{.020} $ 		& $\boldsymbol{19.7 \pm .5}$ 	& $\boldsymbol{20.5 \pm .5} $ & $\boldsymbol{.984}$ \\
	\textbf{Pre-Meiyu} (121-160) 		& $27.5 \pm .3$ & $27.4 \pm .3$ & $ .29 $ 		& $25.4 \pm .7$ 	& $25.6 \pm .8	$ & $.72$ \\
	\textbf{Meiyu} (161-200)			& $29.6 \pm .3$ & $29.4 \pm .3$ & $ .24 $ 		& $28.2 \pm .8$ 	& $28.4 \pm .8	$  & $.71$ \\
	\textbf{Post-Meiyu} (201-273)		& $\boldsymbol{30.2 \pm .3^*}$ & $\boldsymbol{29.6 \pm .3^*}$ & $\boldsymbol{.0048^*} $	& $25.5 \pm .7$ 	& $25.7 \pm .7	$ & $.71$ \\
	\textbf{Post-Meiyu}, $>27^\circ$N 	& $\boldsymbol{33.6 \pm .2^*}$ & $\boldsymbol{32.9 \pm .3^*}$ & $\boldsymbol{.0003^*} $ 	& $23.5 \pm .7$ 	& $24.2 \pm .7	$ & $.92$ \\
	\textbf{Post-Meiyu}, $<27^\circ$N 	& $23.7 \pm .1 $ & $23.8 \pm .2$ & $ .83 $ 	& $29.1 \pm 1.3$ 	& $28.3 \pm 1.4	$ & $.20$  \\
	\textbf{Fall Rains} (274-320)			& $29.1 \pm .4 $ & $29.3 \pm .4$ & $ .79 $ 	& $20.3 \pm 1.0$ 	& $20.8 \pm .9	$ & $.76$ \\
	\textbf{Full Year} (1-365)			& $\boldsymbol{28.7 \pm .1^*}$ & $\boldsymbol{28.5 \pm .1^*}$ & $\boldsymbol{.0032^*}$ 	& $23.3 \pm .3$ 	& $23.6 \pm .3	$ & $.95$ \\

\end{tabular}
\label{ts6}
\end{table}

%% TABLE 7 - change in rainband frequency between 1979-1993 and 1994-2007
\begin{table}
\settablenum{S7}  

\centering

\caption{Change in frequency of primary and secondary rainbands between 1979-1993 and 1994-2007, with standard deviation of mean and p-value of change calculated analytically. Statistically significant changes at the 95\%/99\% level are indicated by bold/asterisk and bold.}

\begin{tabular}{ l c c c c c c}
	& \multicolumn{3}{c}{Primary rainband \%} & \multicolumn{3}{c}{Secondary rainband \%} \\
	\textbf{Period} & '79-'93 & '94-'07 & $p$ & '79-'93 & '94-'07 & $p$ \\
	\hline	
	\textbf{Spring Rains} (60-120)		& $50.0 \pm 2.3$ & $45.4 \pm 2.4$ & $ .087 $ 	& $0.9 \pm .3$ 	& $0.5 \pm .2$ & $.14$ \\
	\textbf{Pre-Meiyu} (121-160) 		& $53.0 \pm 2.9$ & $53.2 \pm 3.0$ & $ .52$ 	& $3.5 \pm .7$ 	& $4.1 \pm .8$ & $.71$ \\
	\textbf{Meiyu} (161-200)			& $63.7 \pm 2.9$ & $64.8 \pm 3.0$ & $ .61 $ 	& $8.7 \pm 1.2$ 	& $9.5 \pm 1.3$  & $.67$ \\
	\textbf{Post-Meiyu} (201-273)		& $41.6 \pm 2.1$ & $42.7 \pm 2.1$ & $ .63 $	& $8.0 \pm 1.0$ 	& $8.0 \pm 1.0$ & $.50$ \\
	\textbf{Post-Meiyu}, $>27^\circ$N 	& $27.2 \pm 1.9$ & $25.2 \pm 1.9$ & $ .23 $ 	& $3.1 \pm .6$ 	& $2.9 \pm .6$ & $.42$ \\
	\textbf{Post-Meiyu}, $<27^\circ$N 	& $14.4 \pm 1.5 $ & $17.4 \pm 1.6$ & $ .91 $ 	& $4.9 \pm .8$ 	& $5.1 \pm .8$ & $.55$  \\
	\textbf{Fall Rains} (274-320)			& $26.4 \pm 2.4 $ & $27.0 \pm 2.5$ & $ .58 $ 	& $1.6 \pm .6$ 	& $0.8 \pm .4$ & $.13$ \\
	\textbf{Full Year} (1-365)			& $37.9 \pm 0.9 $ & $38.2 \pm 0.9$ & $ .59 $ 	& $3.3 \pm .3$ 	& $3.3 \pm .3$ & $.52$ \\

\end{tabular}
\label{ts7}
\end{table}

%% TABLE 8 - change in rainband latitude and intensity between 1979-1993 and 1994-2007
\begin{table}
\settablenum{S8}  

\centering

\caption{Change in latitude and intensity of rainbands between 1979-1993 and 1994-2007, with standard deviation of mean and p-value of change both calculated by a permutation test with 10,000 iterations. Statistically significant changes at the 95\%/99\% level are indicated by bold/asterisk and bold.}

\begin{tabular}{ l c c c c c c}
	& \multicolumn{3}{c}{Rainband latitude ($^\circ$)} & \multicolumn{3}{c}{Intensity (mm day$^{-1})$} \\
	\textbf{Period} & '79-'93 & '94-'07 & $p$ & '79-'93 & '94-'07 & $p$ \\
	\hline	
	\textbf{Spring Rains} (60-120)		& $27.2 \pm .3 $ & $27.5 \pm .3 $ & $ .967 $ 	& $20.5 \pm .7$ 	& $20.6 \pm .8 	$ & $.54$ \\
	\textbf{Pre-Meiyu} (121-160) 		& $27.4 \pm .4 $ & $27.2 \pm .4$ & $ .23 $ 	& $25.0 \pm 1.0$ 	& $26.2 \pm 1.1	$ & $.94$ \\
	\textbf{Meiyu} (161-200)			& $\boldsymbol{30.0 \pm .4^*}$ & $\boldsymbol{28.9 \pm .4^*}$ & $\boldsymbol{.0002^*}$ & $\boldsymbol{27.3 \pm 1.1^*}$ 	& $\boldsymbol{29.8 \pm 1.1^*}$  & $\boldsymbol{.9994 ^*}$ \\
	\textbf{Post-Meiyu} (201-273)		& $29.8 \pm .4 $ & $29.3 \pm .5 $ & $ .092 $	& $25.9 \pm .9$ 	& $25.4 \pm .9	$ & $.28$ \\
	\textbf{Post-Meiyu}, $>27^\circ$N 	& $32.8 \pm .3 $ & $33.0 \pm .4 $ & $ .80 $ 	& $24.4 \pm 1.0$ 	& $23.9 \pm 1.1	$ & $.24$ \\
	\textbf{Post-Meiyu}, $<27^\circ$N 	& $23.8 \pm .2 $ & $23.8 \pm .2 $ & $ .48 $ 	& $28.7 \pm 1.8$ 	& $27.9 \pm 1.7	$ & $.28$  \\
	\textbf{Fall Rains} (274-320)			& $\boldsymbol{28.9 \pm .5} $ & $\boldsymbol{29.7 \pm .6} $ & $ \boldsymbol{.982} $ 	& $20.1 \pm 1.4$ 	& $21.7 \pm 1.4	$ & $.94$ \\
	\textbf{Full Year} (1-365)			& $28.6 \pm .2 $ & $28.4 \pm .2 $ & $ .13 $ 	& $\boldsymbol{23.3 \pm .4}$ 	& $\boldsymbol{24.0 \pm .4}	$ & $\boldsymbol{.982}$ \\

\end{tabular}
\label{ts8}
\end{table}



%%% SUPPLEMENTARY FIGURES %%%
 
%S1 - displaying continuous maximum criterion required to attempt rainband fit
\begin{figure}
\setfigurenum{S1}  

\noindent\includegraphics[width=36pc]{Figures/S1}
\caption{The first step of the rainband detection algorithm checks to see whether a five-degree continuous longitudinal band of precipitation maxima above 10 mm day$^{-1}$ exists. If so, a rainband fit is attempted. a) 25 May 2007 - the continuous maximum criterion is met and a fit is attempted. b) 11 June 2007 - although there is abundant rainfall in some locations, it appears not to be frontal and the continuous maximum criterion is failed. No fit is attempted.}
\end{figure}

%S2 - How the convergent fit algorithm works.
\begin{figure}
\setfigurenum{S2}  

\noindent\includegraphics[width=36pc]{Figures/S2}
\caption{Display of the functionality of the recursive convergent algorithm. On 29 April 2007, a strong maximum in southernmost China skews our initial rainband fit (a), but the algorithm eventually converges on the most prominent coherent band via tighter windowing (d).}
\end{figure}

%S3 - Procedure for finding double rainbands
\begin{figure}
\setfigurenum{S3}  

\noindent\includegraphics[width=36pc]{Figures/S3}
\caption{a) The algorithm converges on the strongest rainband, around 37N (defined as the ``primary rainband''). b) The rainfall associated with the primary rainband is removed, and we check for the presence of another rainband (a ``secondary rainband''), again using the continuous maximum criterion.}
\end{figure}

%S4 - Quality Control algorithm used to determine inclusion in statistics
\begin{figure}
\setfigurenum{S4}  

\noindent\includegraphics[width=36pc]{Figures/S4}
\caption{A quality control algorithm is used to exclude poor fits. a) 18 August 2007 - Days with a high Taiwan fraction (here, corresponding to the passage of Typhoon Sepat) are excluded from our statistics. b) June 4 2007 - A high-quality fit is achieved. c) 17 April 2007 - Although a fit is reached, it explains the distribution of daily rainfall poorly and is therefore excluded from rainband statistics. d) 21 May 2007 (same day as Figure S3) - An initial fit appears to be of poor quality ($Q<.6$). However, after finding a secondary rainband, we determine that conditional quality scores $Q_1$ and $Q_2$ are high, and the day is included in our statistics.}
\end{figure}

%S5 - Autocorrelation of jet during Pre-Meiyu and Post-Meiyu
\begin{figure}
\setfigurenum{S5}  
\centering

\noindent\includegraphics[width=24pc]{Figures/S5}
\caption{Autocorrelation of the jet during the Pre-Meiyu season (days 121-160) for a) daily and b) 4-day blocks; and during the Post-Meiyu season (days 201-273) for c) daily and d) 7-day blocks. Block length is longer during Post-Meiyu due to longer autocorrelation time scale of mean jet latitude. The compiling of daily mean jet data into longer blocks removes temporal autocorrelation, allowing for the use of the bootstrap Kolmogorov-Smirnov test to compare changes in mean latitude between two time periods.}
\end{figure}


\begin{thebibliography}{2}
\providecommand{\natexlab}[1]{#1}
\expandafter\ifx\csname urlstyle\endcsname\relax
  \providecommand{\doi}[1]{doi:\discretionary{}{}{}#1}\else
  \providecommand{\doi}{doi:\discretionary{}{}{}\begingroup
  \urlstyle{rm}\Url}\fi

\bibitem[{\textit{Schiemann et~al.}(2009)\textit{Schiemann, L\"{u}thi, and
  Sch\"{a}r}}]{Schiemann2009}
Schiemann, R., D.~L\"{u}thi, and C.~Sch\"{a}r (2009), {Seasonality and
  Interannual Variability of the Westerly Jet in the Tibetan Plateau Region},
  \textit{J. Clim.}, \textit{22}, 2940--2957, \doi{10.1175/2008JCLI2625.1}.

\bibitem[{\textit{Yatagai et~al.}(2012)\textit{Yatagai, Kamiguchi, Arakawa,
  Hamada, Yasutomi, and Kitoh}}]{Yatagai2012}
Yatagai, A., K.~Kamiguchi, O.~Arakawa, A.~Hamada, N.~Yasutomi, and A.~Kitoh
  (2012), {APHRODITE: Constructing a Long-Term Daily Gridded Precipitation
  Dataset for Asia Based on a Dense Network of Rain Gauges}, \textit{Bull. Am.
  Meteorol. Soc.}, \textit{93}, 1401--1415, \doi{10.1175/BAMS-D-11-00122.1}.

\end{thebibliography}

\end{document}
