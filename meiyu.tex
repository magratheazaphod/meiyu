%%%%%%%%%%%%%%%%%%%%%%%%%%%%%%%%%%%%%%%%%%%%%%%%%%%%%%%%%%%%%%%%%%%%%%%%%%%%
% AGUtmpl.tex: this template file is for articles formatted with LaTeX2e,
% Modified March 2013
%
% This template includes commands and instructions
% given in the order necessary to produce a final output that will
% satisfy AGU requirements.
%
% PLEASE DO NOT USE YOUR OWN MACROS
% DO NOT USE \newcommand, \renewcommand, or \def.
%
% FOR FIGURES, DO NOT USE \psfrag or \subfigure.
%
%%%%%%%%%%%%%%%%%%%%%%%%%%%%%%%%%%%%%%%%%%%%%%%%%%%%%%%%%%%%%%%%%%%%%%%%%%%%
%
% All questions should be e-mailed to latex@agu.org.
%
%%%%%%%%%%%%%%%%%%%%%%%%%%%%%%%%%%%%%%%%%%%%%%%%%%%%%%%%%%%%%%%%%%%%%%%%%%%%
%
% Step 1: Set the \documentclass
%
% There are two options for article format: two column (default)
% and draft.
%
% PLEASE USE THE DRAFT OPTION TO SUBMIT YOUR PAPERS.
% The draft option produces double spaced output.
%
% Choose the journal abbreviation for the journal you are
% submitting to:

% jgrga JOURNAL OF GEOPHYSICAL RESEARCH
% gbc   GLOBAL BIOCHEMICAL CYCLES
% grl   GEOPHYSICAL RESEARCH LETTERS
% pal   PALEOCEANOGRAPHY
% ras   RADIO SCIENCE
% rog   REVIEWS OF GEOPHYSICS
% tec   TECTONICS
% wrr   WATER RESOURCES RESEARCH
% gc    GEOCHEMISTRY, GEOPHYSICS, GEOSYSTEMS
% sw    SPACE WEATHER
% ms    JAMES
% ef    EARTH'S FUTURE
%
%
%
% (If you are submitting to a journal other than jgrga,
% substitute the initials of the journal for "jgrga" below.)

\documentclass[draft,grl]{AGUTeX}
% To create numbered lines:

% If you don't already have lineno.sty, you can download it from
% http://www.ctan.org/tex-archive/macros/latex/contrib/ednotes/
% (or search the internet for lineno.sty ctan), available at TeX Archive Network (CTAN).
% Take care that you always use the latest version.

% To activate the commands, uncomment \usepackage{lineno}
% and \linenumbers*[1]command, below:

\usepackage{lineno}
\linenumbers*[1]

%  To add line numbers to lines with equations:
%  \begin{linenomath*}
%  \begin{equation}
%  \end{equation}
%  \end{linenomath*}
%%%%%%%%%%%%%%%%%%%%%%%%%%%%%%%%%%%%%%%%%%%%%%%%%%%%%%%%%%%%%%%%%%%%%%%%%
% Figures and Tables
%
%
% DO NOT USE \psfrag or \subfigure commands.
%
%  Figures and tables should be placed AT THE END OF THE ARTICLE,
%  after the references.
%
%  Uncomment the following command to include .eps files
%  (comment out this line for draft format):
%  \usepackage[dvips]{graphicx}
%
%  Uncomment the following command to allow illustrations to print
%   when using Draft:
%  \setkeys{Gin}{draft=false}
%
% Substitute one of the following for [dvips] above
% if you are using a different driver program and want to
% proof your illustrations on your machine:
%
% [xdvi], [dvipdf], [dvipsone], [dviwindo], [emtex], [dviwin],
% [pctexps],  [pctexwin],  [pctexhp],  [pctex32], [truetex], [tcidvi],
% [oztex], [textures]
%
% See how to enter figures and tables at the end of the article, after
% references.
%
%% ------------------------------------------------------------------------ %%
%
%  ENTER PREAMBLE
%
%% ------------------------------------------------------------------------ %%

% Author names in capital letters:
\authorrunninghead{DAY ET AL.}

% Shorter version of title entered in capital letters:
\titlerunninghead{COUPLED MEIYU AND JET VARIABILITY}

%Corresponding author mailing address and e-mail address:
\authoraddr{Corresponding author: Jesse Day, University of California Berkeley, Department of Earth and Planetary Science, College of Letters and Science; 307 McCone Hall, Berkeley, CA 94720, USA. (jessed@berkeley.edu)}

\begin{document}

%% ------------------------------------------------------------------------ %%
%
%  TITLE
%
%% ------------------------------------------------------------------------ %%


\title{Covariance of Meiyu Front and Tropospheric Jet Variability on Daily and Interannual time scales}

%% ------------------------------------------------------------------------ %%
%
%  AUTHORS AND AFFILIATIONS
%
%% ------------------------------------------------------------------------ %%


%Use \author{\altaffilmark{}} and \altaffiltext{}

% \altaffilmark will produce footnote;
% matching \altaffiltext will appear at bottom of page.

\authors{Jesse A. Day,\altaffilmark{1}
Jacob Edman,\altaffilmark{1} Inez Fung \altaffilmark{1}, and
Weihan Liu\altaffilmark{1}}

\altaffiltext{1}{Department of Earth and Planetary Science, University of California Berkeley, Berkeley, California, USA.}
%% ------------------------------------------------------------------------ %%
%
%  ABSTRACT
%
%% ------------------------------------------------------------------------ %%

% >> Do NOT include any \begin...\end commands within
% >> the body of the abstract.

\begin{abstract}
This abstract must be 150 words or less.
\end{abstract}

%% ------------------------------------------------------------------------ %%
%
%  BEGIN ARTICLE
%
%% ------------------------------------------------------------------------ %%

% The body of the article must start with a \begin{article} command
%
% \end{article} must follow the references section, before the figures
%  and tables.

\begin{article}

%% ------------------------------------------------------------------------ %%
%
%  TEXT
%
%% ------------------------------------------------------------------------ %%

\section{Introduction}
China receives about 60\% of its rainfall from May to August, a phenomenon referred to as the East Asian Summer Monsoon. Regional peak rates occur from the end of May to the middle of July, when precipitation occurs in continuous frontal bands induced by the Tibetan Plateau upstream. This feature is known as the Meiyu Front, and the duration of its appearance as Meiyu Season. In the annual mean, the Meiyu Front has been claimed to show northward progression and abrupt transitions between preferred latitudes\citep{Ding2005}. Anecdotal evidence suggests an abrupt shift in rainfall patterns beginning in the 1970s, with Northern China experiencing severe droughts and Southern China flooding (``North Dry South Wet''), leading the Chinese government to embark on one of the most expensive engineering products in the history of mankind, the South-North Water Transfer Project. In spite of attempts to attribute observed change to global warming, no mechanism has been agreed on.

Regional prediction of climate change under global warming presents greater difficulty than global projection. In the 5th edition of the IPCC report, the CMIP 5 model suite does not come to a consensus on the sign of future summer rainfall changes in East Asia (ADD CITATION). Several authors have proposed templates for regional mechanisms resulting from CO2 forcing. Radiative constraints on precipitation allow the separation of two terms, global mean response and local circulation effects\citep{Muller2011}. The``rich get richer'' mechanism anticipates increased rainfall in regions of net precipitation and decreases in regions of net evaporation due to amplified moisture transport\citep{Held2006}. Lintner and Neelin\citet{Lintner2007} and Chou et al.\citet{Chou2009} proposed a more comprehensive set of phenomena based on model projection of changes in convective regions. These include not only the``rich get richer'' but also the ``upped ante'' mechanism, wherein convective margins see droughts because increased humidity in convective regions raises the threshold for convection and moisture gradients are stronger. This framework has been used to understand ENSO-related rainfall variability in South America. However, it is difficult to apply these existing theories to a region with high spatial and temporal heterogeneity such as East Asia.

A new paleoclimate study proposes the tropospheric jet as an indicator of past rainfall patterns in China \citep{Nagashima2011}\citep{Nagashima2013}. The authors study a marine sediment core in the Sea of Japan, downstream from both the Taklamakan Desert and Gobi Desert, and are able to differentiate between dust from each of these sources using electron spin resonance (ESR) and grain size. In the present day, Gobi Desert dust is only advected during spring before the tropospheric jet passes north of the Tibetan Plateau and \citep{Roe2009}, whereas the mechanism of transport of Taklamakan Desert dust remains active in summer when the jet occupies a low variability position on the northern flank of the Tibetan Plateau. Therefore, they attribute increases in Gobi Desert dust to longer springs and shorter summers. Since Holocene changes in precipitation match the timing of abrupt changes in their record, they therefore conclude that the tropospheric jet controls precipitation variability over millennial time scales.

The present work aims to test this apparent coupling of the jet and Meiyu Front in the present-day. Current theory suggests that the tropospheric jet plays a major role in Meiyu formation, either by zonal advection of sensible heat from the Tibetan Plateau upwind \citep{Sampe2010}, or as part of the orographically forced circulation that produces meridional wind convergence over China\citep{Chen2014}. Past work has compared jet and Meiyu variability over shorter time periods or with coarse resolution\cite{Liang1998}, but none has systematically performed a comparison with daily data. Since the behavior of the tropospheric jet is coupled to global climate variability, our work holds the promise of attributing rainfall trends in China to global change via the jet.

The climatology of the Meiyu Front has been studied\citep{Ding2005} but no full catalog of interannual and daily variability has previously existed. We use 57 years of rain gauge data over China at .25 by .25 degree resolution\citep{Yatagai2012}. These data were processed with a Meiyu detection algorithm. Our algorithm uses a convergent algorithm to detect continuous zonal precipitation structure and returns information about whether a Meiyu Front is visible on each day, as well as the position, meridional tilt and intensity if a front exists. Poor fits are isolated by using a quality score Q which measures the percentage of rainfall occurring within 300 km of our attempted fit. Our method shows good preliminary ability to reproduce known properties of the jet and northward progression during Meiyu Season. 

For tropospheric jet variability, we employ a database based on ERA-40 reanalysis data developed by Schiemann et al.\citet{Schiemann2009}. Their database includes every appearance of a tropospheric jet in East Asia for 1958-2001 at 6-hourly intervals using simple criteria: Positive zonal wind and local maximum in excess of 30 m/s.

	We first attempt to define a transition date from spring to summer behavior in the jet database, and equivalently from Meiyu Season to post-Meiyu in our new catalog. Preliminary evidence suggests a long-term perturbation in mean jet path in East Asia from the 1960s to present with later onset of summer jet and shorter total duration of summer jet. In our Meiyu database it is more difficult to extract an exact transition date due to high-frequency variability in space and time. However, we observe an apparent shift in the timing of northward progression of the Meiyu Front between 1951-1970 and 1988-2007. If both databases demonstrate a robust decadal shift, they may provide an explanation for the anecdotal South Wet-North Dry pattern of rainfall change.

	Finally, we use our knowledge of daily Meiyu positions to isolate preferred configurations for different dates, as well as probability distributions of the tropospheric jet associated with each configuration. If a robust change in mean jet progression is detected, we may be able to isolate a corresponding shift in Meiyu distribution that may have previously gone unnoticed due to extreme temporal variability in the data.

	In the following text we seek to achieve the following objectives: 1) Define an objective climatology of the Meiyu front; 2) Show the positions of the tropospheric jet associated with different configurations of the tropospheric jet; 3) Show that the Meiyu front and tropospheric jet covary on a daily scale; and 4) Suggest a dynamic lens to understand the observed climatology of Meiyu front and jet.
	
\section{Data sets}
The analysis contained in this paper relies on two data sets, APHRODITE and Schiemann et al's database of jet counts. APHRODITE (Asian Precipitation - Highly-Resolved Observational Data Integration Towards Evaluation of the Water Resources) \citep{Yatagai2012}. The APHRO\_MA\_V1101 product includes 57 years (1951-2007) of daily precipitation (PRECIP product, units mm day$^{-1}$) and station coverage (RSTN product) on a .25\textdegree\ $\times$ .25\textdegree\ grid (roughly 25 km spacing) between 60\textdegree E-150\textdegree E and 15\textdegree S-55\textdegree N.
	
\section{An ``objective'' Meiyu climatology}
\subsection{Algorithm}
	
\subsection{Defining the length of Meiyu season}	
	
\section{Preferred jet positions}
	
\section{Covariance of jet and Meiyu anomalies}
	
\section{Dynamics}

	The covariance of Meiyu front positions and tropospheric jet latitudes previously demonstrated also clarifies a dynamical reason for their seasonality. As shown, Meiyu season consists of intense rainfall from June 1st to July 1st with a shift in latitude of almost 10 degrees over the course of that month. After . The lens of forced convergence by the Tibetan Plateau. In the climatological mean, \cite{Chen2014} demonstrated that the Meiyu front exists primarily due to forced mechanical convergence by the Tibetan Plateau upstream. They showed this by using experiments in which the Tibetan Plateau's height was reduced by 95\%. We argue that our results similarly show the role of the Tibetan Plateau in generating a strong Meiyu front. When the jet impinges on the Tibetan Plateau from late May to early July, the high topography induces meanders that force standing waves in jet configuration, pushing it further north from 80E to 100E and then further south into China. This in turn anchors strong rainfall along the Meiyu front in China. When the jet moves further north to its preferred summer position, which occurs just north of the Tibetan Plateau, it is no longer deflected and the front is significantly weaker, though events do still occur as seen in Figure 1. We confirm this hypothesis by showing the climatology of precipitable water vapor from SSRI from time A, time B, time C and time D (last figure). During the early and late stages of Meiyu season precipitable water content is concentrated along bands that intersect central China. However, in July and August, the latitude of the moisture vapor front has shifted much further north, over northern China, and both the Bay of Bohai and the Korean Peninsula and Japan all have much greater precipitable water. However, the lack of mechanical forcing is shown by the weakness of rainfall in those events (\lessthan 20 mm/day versus 25-30 mm/day over southern and central China during Meiyu season.

%%% End of body of article:
%  ACKNOWLEDGMENTS

\begin{acknowledgments}
APHRODITE is... 
\end{acknowledgments}

%% ------------------------------------------------------------------------ %%
%%  REFERENCE LIST AND TEXT CITATIONS
%
% Either type in your references using
% \begin{thebibliography}{}
% \bibitem{}
% Text
% \end{thebibliography}
%
% Or,
%
% If you use BiBTeX for your references, please use the agufull08.bst file (available at % ftp://ftp.agu.org/journals/latex/journals/Manuscript-Preparation/) to produce your .bbl
% file and copy the contents into your paper here.
%
% Follow these steps:
% 1. Run LaTeX on your LaTeX file.
%
% 2. Make sure the bibliography style appears as \bibliographystyle{agufull08}. Run BiBTeX on your LaTeX
% file.
%
% 3. Open the new .bbl file containing the reference list and
%   copy all the contents into your LaTeX file here.
%
% 4. Comment out the old \bibliographystyle and \bibliography commands.
%
% 5. Run LaTeX on your new file before submitting.
%
% AGU does not want a .bib or a .bbl file. Please copy in the contents of your .bbl file here.

\bibliographystyle{agufull08}
\bibliography{"/Users/Siwen/Desktop/Spring 2014/meiyu"}

%% ------------------------------------------------------------------------ %%
%
%  END ARTICLE
%
%% ------------------------------------------------------------------------ %%
\end{article}
%
%
%% Enter Figures and Tables here:
%
% DO NOT USE \psfrag or \subfigure commands.
%
% Figure captions go below the figure.
% Table titles go above tables; all other caption information
%  should be placed in footnotes below the table.
%
%----------------
% EXAMPLE FIGURE
%
% \begin{figure}
% \noindent\includegraphics[width=20pc]{samplefigure.eps}
% \caption{Caption text here}
% \label{figure_label}
% \end{figure}

\end{document}

%%%%%%%%%%%%%%%%%%%%%%%%%%%%%%%%%%%%%%%%%%%%%%%%%%%%%%%%%%%%%%%

More Information and Advice:

%% ------------------------------------------------------------------------ %%
%
%  SECTION HEADS
%
%% ------------------------------------------------------------------------ %%

% Capitalize the first letter of each word (except for
% prepositions, conjunctions, and articles that are
% three or fewer letters).

% AGU follows standard outline style; therefore, there cannot be a section 1 without
% a section 2, or a section 2.3.1 without a section 2.3.2.
% Please make sure your section numbers are balanced.
% ---------------
% Level 1 head
%
% Use the \section{} command to identify level 1 heads;
% type the appropriate head wording between the curly
% brackets, as shown below.
%
%An example:
%\section{Level 1 Head: Introduction}
%
% ---------------
% Level 2 head
%
% Use the \subsection{} command to identify level 2 heads.
%An example:
%\subsection{Level 2 Head}
%
% ---------------
% Level 3 head
%
% Use the \subsubsection{} command to identify level 3 heads
%An example:
%\subsubsection{Level 3 Head}
%
%---------------
% Level 4 head
%
% Use the \subsubsubsection{} command to identify level 3 heads
% An example:
%\subsubsubsection{Level 4 Head} An example.
%
%% ------------------------------------------------------------------------ %%
%
%  IN-TEXT LISTS
%
%% ------------------------------------------------------------------------ %%
%
% Do not use bulleted lists; enumerated lists are okay.
% \begin{enumerate}
% \item
% \item
% \item
% \end{enumerate}
%
%% ------------------------------------------------------------------------ %%
%
%  EQUATIONS
%
%% ------------------------------------------------------------------------ %%

% Single-line equations are centered.
% Equation arrays will appear left-aligned.

Math coded inside display math mode \[ ...\]
 will not be numbered, e.g.,:
 \[ x^2=y^2 + z^2\]

 Math coded inside \begin{equation} and \end{equation} will
 be automatically numbered, e.g.,:
 \begin{equation}
 x^2=y^2 + z^2
 \end{equation}

% IF YOU HAVE MULTI-LINE EQUATIONS, PLEASE
% BREAK THE EQUATIONS INTO TWO OR MORE LINES
% OF SINGLE COLUMN WIDTH (20 pc, 8.3 cm)
% using double backslashes (\\).

% To create multiline equations, use the
% \begin{eqnarray} and \end{eqnarray} environment
% as demonstrated below.
\begin{eqnarray}
  x_{1} & = & (x - x_{0}) \cos \Theta \nonumber \\
        && + (y - y_{0}) \sin \Theta  \nonumber \\
  y_{1} & = & -(x - x_{0}) \sin \Theta \nonumber \\
        && + (y - y_{0}) \cos \Theta.
\end{eqnarray}

%If you don't want an equation number, use the star form:
%\begin{eqnarray*}...\end{eqnarray*}

% Break each line at a sign of operation
% (+, -, etc.) if possible, with the sign of operation
% on the new line.

% Indent second and subsequent lines to align with
% the first character following the equal sign on the
% first line.

% Use an \hspace{} command to insert horizontal space
% into your equation if necessary. Place an appropriate
% unit of measure between the curly braces, e.g.
% \hspace{1in}; you may have to experiment to achieve
% the correct amount of space.


%% ------------------------------------------------------------------------ %%
%
%  EQUATION NUMBERING: COUNTER
%
%% ------------------------------------------------------------------------ %%

% You may change equation numbering by resetting
% the equation counter or by explicitly numbering
% an equation.

% To explicitly number an equation, type \eqnum{}
% (with the desired number between the brackets)
% after the \begin{equation} or \begin{eqnarray}
% command.  The \eqnum{} command will affect only
% the equation it appears with; LaTeX will number
% any equations appearing later in the manuscript
% according to the equation counter.
%

% If you have a multiline equation that needs only
% one equation number, use a \nonumber command in
% front of the double backslashes (\\) as shown in
% the multiline equation above.

%% ------------------------------------------------------------------------ %%
%
%  SIDEWAYS FIGURE AND TABLE EXAMPLES
%
%% ------------------------------------------------------------------------ %%
%
% For tables and figures, add \usepackage{rotating} to the paper and add the rotating.sty file to the folder.
% AGU prefers the use of {sidewaystable} over {landscapetable} as it causes fewer problems.
%
% \begin{sidewaysfigure}
% \includegraphics[width=20pc]{samplefigure.eps}
% \caption{caption here}
% \label{label_here}
% \end{sidewaysfigure}
%
%
%
% \begin{sidewaystable}
% \caption{}
% \begin{tabular}
% Table layout here.
% \end{tabular}
% \end{sidewaystable}
%
%

